\documentclass[11pt]{article}

    \usepackage[breakable]{tcolorbox}
    \usepackage{parskip} % Stop auto-indenting (to mimic markdown behaviour)
    
    \usepackage{iftex}
    \ifPDFTeX
    	\usepackage[T1]{fontenc}
    	\usepackage{mathpazo}
    \else
    	\usepackage{fontspec}
    \fi

    % Basic figure setup, for now with no caption control since it's done
    % automatically by Pandoc (which extracts ![](path) syntax from Markdown).
    \usepackage{graphicx}
    % Maintain compatibility with old templates. Remove in nbconvert 6.0
    \let\Oldincludegraphics\includegraphics
    % Ensure that by default, figures have no caption (until we provide a
    % proper Figure object with a Caption API and a way to capture that
    % in the conversion process - todo).
    \usepackage{caption}
    \DeclareCaptionFormat{nocaption}{}
    \captionsetup{format=nocaption,aboveskip=0pt,belowskip=0pt}

    \usepackage[Export]{adjustbox} % Used to constrain images to a maximum size
    \adjustboxset{max size={0.9\linewidth}{0.9\paperheight}}
    \usepackage{float}
    \floatplacement{figure}{H} % forces figures to be placed at the correct location
    \usepackage{xcolor} % Allow colors to be defined
    \usepackage{enumerate} % Needed for markdown enumerations to work
    \usepackage{geometry} % Used to adjust the document margins
    \usepackage{amsmath} % Equations
    \usepackage{amssymb} % Equations
    \usepackage{textcomp} % defines textquotesingle
    % Hack from http://tex.stackexchange.com/a/47451/13684:
    \AtBeginDocument{%
        \def\PYZsq{\textquotesingle}% Upright quotes in Pygmentized code
    }
    \usepackage{upquote} % Upright quotes for verbatim code
    \usepackage{eurosym} % defines \euro
    \usepackage[mathletters]{ucs} % Extended unicode (utf-8) support
    \usepackage{fancyvrb} % verbatim replacement that allows latex
    \usepackage{grffile} % extends the file name processing of package graphics 
                         % to support a larger range
    \makeatletter % fix for grffile with XeLaTeX
    \def\Gread@@xetex#1{%
      \IfFileExists{"\Gin@base".bb}%
      {\Gread@eps{\Gin@base.bb}}%
      {\Gread@@xetex@aux#1}%
    }
    \makeatother

    % The hyperref package gives us a pdf with properly built
    % internal navigation ('pdf bookmarks' for the table of contents,
    % internal cross-reference links, web links for URLs, etc.)
    \usepackage{hyperref}
    % The default LaTeX title has an obnoxious amount of whitespace. By default,
    % titling removes some of it. It also provides customization options.
    \usepackage{titling}
    \usepackage{longtable} % longtable support required by pandoc >1.10
    \usepackage{booktabs}  % table support for pandoc > 1.12.2
    \usepackage[inline]{enumitem} % IRkernel/repr support (it uses the enumerate* environment)
    \usepackage[normalem]{ulem} % ulem is needed to support strikethroughs (\sout)
                                % normalem makes italics be italics, not underlines
    \usepackage{mathrsfs}
    

    
    % Colors for the hyperref package
    \definecolor{urlcolor}{rgb}{0,.145,.698}
    \definecolor{linkcolor}{rgb}{.71,0.21,0.01}
    \definecolor{citecolor}{rgb}{.12,.54,.11}

    % ANSI colors
    \definecolor{ansi-black}{HTML}{3E424D}
    \definecolor{ansi-black-intense}{HTML}{282C36}
    \definecolor{ansi-red}{HTML}{E75C58}
    \definecolor{ansi-red-intense}{HTML}{B22B31}
    \definecolor{ansi-green}{HTML}{00A250}
    \definecolor{ansi-green-intense}{HTML}{007427}
    \definecolor{ansi-yellow}{HTML}{DDB62B}
    \definecolor{ansi-yellow-intense}{HTML}{B27D12}
    \definecolor{ansi-blue}{HTML}{208FFB}
    \definecolor{ansi-blue-intense}{HTML}{0065CA}
    \definecolor{ansi-magenta}{HTML}{D160C4}
    \definecolor{ansi-magenta-intense}{HTML}{A03196}
    \definecolor{ansi-cyan}{HTML}{60C6C8}
    \definecolor{ansi-cyan-intense}{HTML}{258F8F}
    \definecolor{ansi-white}{HTML}{C5C1B4}
    \definecolor{ansi-white-intense}{HTML}{A1A6B2}
    \definecolor{ansi-default-inverse-fg}{HTML}{FFFFFF}
    \definecolor{ansi-default-inverse-bg}{HTML}{000000}

    % commands and environments needed by pandoc snippets
    % extracted from the output of `pandoc -s`
    \providecommand{\tightlist}{%
      \setlength{\itemsep}{0pt}\setlength{\parskip}{0pt}}
    \DefineVerbatimEnvironment{Highlighting}{Verbatim}{commandchars=\\\{\}}
    % Add ',fontsize=\small' for more characters per line
    \newenvironment{Shaded}{}{}
    \newcommand{\KeywordTok}[1]{\textcolor[rgb]{0.00,0.44,0.13}{\textbf{{#1}}}}
    \newcommand{\DataTypeTok}[1]{\textcolor[rgb]{0.56,0.13,0.00}{{#1}}}
    \newcommand{\DecValTok}[1]{\textcolor[rgb]{0.25,0.63,0.44}{{#1}}}
    \newcommand{\BaseNTok}[1]{\textcolor[rgb]{0.25,0.63,0.44}{{#1}}}
    \newcommand{\FloatTok}[1]{\textcolor[rgb]{0.25,0.63,0.44}{{#1}}}
    \newcommand{\CharTok}[1]{\textcolor[rgb]{0.25,0.44,0.63}{{#1}}}
    \newcommand{\StringTok}[1]{\textcolor[rgb]{0.25,0.44,0.63}{{#1}}}
    \newcommand{\CommentTok}[1]{\textcolor[rgb]{0.38,0.63,0.69}{\textit{{#1}}}}
    \newcommand{\OtherTok}[1]{\textcolor[rgb]{0.00,0.44,0.13}{{#1}}}
    \newcommand{\AlertTok}[1]{\textcolor[rgb]{1.00,0.00,0.00}{\textbf{{#1}}}}
    \newcommand{\FunctionTok}[1]{\textcolor[rgb]{0.02,0.16,0.49}{{#1}}}
    \newcommand{\RegionMarkerTok}[1]{{#1}}
    \newcommand{\ErrorTok}[1]{\textcolor[rgb]{1.00,0.00,0.00}{\textbf{{#1}}}}
    \newcommand{\NormalTok}[1]{{#1}}
    
    % Additional commands for more recent versions of Pandoc
    \newcommand{\ConstantTok}[1]{\textcolor[rgb]{0.53,0.00,0.00}{{#1}}}
    \newcommand{\SpecialCharTok}[1]{\textcolor[rgb]{0.25,0.44,0.63}{{#1}}}
    \newcommand{\VerbatimStringTok}[1]{\textcolor[rgb]{0.25,0.44,0.63}{{#1}}}
    \newcommand{\SpecialStringTok}[1]{\textcolor[rgb]{0.73,0.40,0.53}{{#1}}}
    \newcommand{\ImportTok}[1]{{#1}}
    \newcommand{\DocumentationTok}[1]{\textcolor[rgb]{0.73,0.13,0.13}{\textit{{#1}}}}
    \newcommand{\AnnotationTok}[1]{\textcolor[rgb]{0.38,0.63,0.69}{\textbf{\textit{{#1}}}}}
    \newcommand{\CommentVarTok}[1]{\textcolor[rgb]{0.38,0.63,0.69}{\textbf{\textit{{#1}}}}}
    \newcommand{\VariableTok}[1]{\textcolor[rgb]{0.10,0.09,0.49}{{#1}}}
    \newcommand{\ControlFlowTok}[1]{\textcolor[rgb]{0.00,0.44,0.13}{\textbf{{#1}}}}
    \newcommand{\OperatorTok}[1]{\textcolor[rgb]{0.40,0.40,0.40}{{#1}}}
    \newcommand{\BuiltInTok}[1]{{#1}}
    \newcommand{\ExtensionTok}[1]{{#1}}
    \newcommand{\PreprocessorTok}[1]{\textcolor[rgb]{0.74,0.48,0.00}{{#1}}}
    \newcommand{\AttributeTok}[1]{\textcolor[rgb]{0.49,0.56,0.16}{{#1}}}
    \newcommand{\InformationTok}[1]{\textcolor[rgb]{0.38,0.63,0.69}{\textbf{\textit{{#1}}}}}
    \newcommand{\WarningTok}[1]{\textcolor[rgb]{0.38,0.63,0.69}{\textbf{\textit{{#1}}}}}
    
    
    % Define a nice break command that doesn't care if a line doesn't already
    % exist.
    \def\br{\hspace*{\fill} \\* }
    % Math Jax compatibility definitions
    \def\gt{>}
    \def\lt{<}
    \let\Oldtex\TeX
    \let\Oldlatex\LaTeX
    \renewcommand{\TeX}{\textrm{\Oldtex}}
    \renewcommand{\LaTeX}{\textrm{\Oldlatex}}
    % Document parameters
    % Document title
    \title{adversarial\_fgsm}
    
    
    
    
    
% Pygments definitions
\makeatletter
\def\PY@reset{\let\PY@it=\relax \let\PY@bf=\relax%
    \let\PY@ul=\relax \let\PY@tc=\relax%
    \let\PY@bc=\relax \let\PY@ff=\relax}
\def\PY@tok#1{\csname PY@tok@#1\endcsname}
\def\PY@toks#1+{\ifx\relax#1\empty\else%
    \PY@tok{#1}\expandafter\PY@toks\fi}
\def\PY@do#1{\PY@bc{\PY@tc{\PY@ul{%
    \PY@it{\PY@bf{\PY@ff{#1}}}}}}}
\def\PY#1#2{\PY@reset\PY@toks#1+\relax+\PY@do{#2}}

\expandafter\def\csname PY@tok@w\endcsname{\def\PY@tc##1{\textcolor[rgb]{0.73,0.73,0.73}{##1}}}
\expandafter\def\csname PY@tok@c\endcsname{\let\PY@it=\textit\def\PY@tc##1{\textcolor[rgb]{0.25,0.50,0.50}{##1}}}
\expandafter\def\csname PY@tok@cp\endcsname{\def\PY@tc##1{\textcolor[rgb]{0.74,0.48,0.00}{##1}}}
\expandafter\def\csname PY@tok@k\endcsname{\let\PY@bf=\textbf\def\PY@tc##1{\textcolor[rgb]{0.00,0.50,0.00}{##1}}}
\expandafter\def\csname PY@tok@kp\endcsname{\def\PY@tc##1{\textcolor[rgb]{0.00,0.50,0.00}{##1}}}
\expandafter\def\csname PY@tok@kt\endcsname{\def\PY@tc##1{\textcolor[rgb]{0.69,0.00,0.25}{##1}}}
\expandafter\def\csname PY@tok@o\endcsname{\def\PY@tc##1{\textcolor[rgb]{0.40,0.40,0.40}{##1}}}
\expandafter\def\csname PY@tok@ow\endcsname{\let\PY@bf=\textbf\def\PY@tc##1{\textcolor[rgb]{0.67,0.13,1.00}{##1}}}
\expandafter\def\csname PY@tok@nb\endcsname{\def\PY@tc##1{\textcolor[rgb]{0.00,0.50,0.00}{##1}}}
\expandafter\def\csname PY@tok@nf\endcsname{\def\PY@tc##1{\textcolor[rgb]{0.00,0.00,1.00}{##1}}}
\expandafter\def\csname PY@tok@nc\endcsname{\let\PY@bf=\textbf\def\PY@tc##1{\textcolor[rgb]{0.00,0.00,1.00}{##1}}}
\expandafter\def\csname PY@tok@nn\endcsname{\let\PY@bf=\textbf\def\PY@tc##1{\textcolor[rgb]{0.00,0.00,1.00}{##1}}}
\expandafter\def\csname PY@tok@ne\endcsname{\let\PY@bf=\textbf\def\PY@tc##1{\textcolor[rgb]{0.82,0.25,0.23}{##1}}}
\expandafter\def\csname PY@tok@nv\endcsname{\def\PY@tc##1{\textcolor[rgb]{0.10,0.09,0.49}{##1}}}
\expandafter\def\csname PY@tok@no\endcsname{\def\PY@tc##1{\textcolor[rgb]{0.53,0.00,0.00}{##1}}}
\expandafter\def\csname PY@tok@nl\endcsname{\def\PY@tc##1{\textcolor[rgb]{0.63,0.63,0.00}{##1}}}
\expandafter\def\csname PY@tok@ni\endcsname{\let\PY@bf=\textbf\def\PY@tc##1{\textcolor[rgb]{0.60,0.60,0.60}{##1}}}
\expandafter\def\csname PY@tok@na\endcsname{\def\PY@tc##1{\textcolor[rgb]{0.49,0.56,0.16}{##1}}}
\expandafter\def\csname PY@tok@nt\endcsname{\let\PY@bf=\textbf\def\PY@tc##1{\textcolor[rgb]{0.00,0.50,0.00}{##1}}}
\expandafter\def\csname PY@tok@nd\endcsname{\def\PY@tc##1{\textcolor[rgb]{0.67,0.13,1.00}{##1}}}
\expandafter\def\csname PY@tok@s\endcsname{\def\PY@tc##1{\textcolor[rgb]{0.73,0.13,0.13}{##1}}}
\expandafter\def\csname PY@tok@sd\endcsname{\let\PY@it=\textit\def\PY@tc##1{\textcolor[rgb]{0.73,0.13,0.13}{##1}}}
\expandafter\def\csname PY@tok@si\endcsname{\let\PY@bf=\textbf\def\PY@tc##1{\textcolor[rgb]{0.73,0.40,0.53}{##1}}}
\expandafter\def\csname PY@tok@se\endcsname{\let\PY@bf=\textbf\def\PY@tc##1{\textcolor[rgb]{0.73,0.40,0.13}{##1}}}
\expandafter\def\csname PY@tok@sr\endcsname{\def\PY@tc##1{\textcolor[rgb]{0.73,0.40,0.53}{##1}}}
\expandafter\def\csname PY@tok@ss\endcsname{\def\PY@tc##1{\textcolor[rgb]{0.10,0.09,0.49}{##1}}}
\expandafter\def\csname PY@tok@sx\endcsname{\def\PY@tc##1{\textcolor[rgb]{0.00,0.50,0.00}{##1}}}
\expandafter\def\csname PY@tok@m\endcsname{\def\PY@tc##1{\textcolor[rgb]{0.40,0.40,0.40}{##1}}}
\expandafter\def\csname PY@tok@gh\endcsname{\let\PY@bf=\textbf\def\PY@tc##1{\textcolor[rgb]{0.00,0.00,0.50}{##1}}}
\expandafter\def\csname PY@tok@gu\endcsname{\let\PY@bf=\textbf\def\PY@tc##1{\textcolor[rgb]{0.50,0.00,0.50}{##1}}}
\expandafter\def\csname PY@tok@gd\endcsname{\def\PY@tc##1{\textcolor[rgb]{0.63,0.00,0.00}{##1}}}
\expandafter\def\csname PY@tok@gi\endcsname{\def\PY@tc##1{\textcolor[rgb]{0.00,0.63,0.00}{##1}}}
\expandafter\def\csname PY@tok@gr\endcsname{\def\PY@tc##1{\textcolor[rgb]{1.00,0.00,0.00}{##1}}}
\expandafter\def\csname PY@tok@ge\endcsname{\let\PY@it=\textit}
\expandafter\def\csname PY@tok@gs\endcsname{\let\PY@bf=\textbf}
\expandafter\def\csname PY@tok@gp\endcsname{\let\PY@bf=\textbf\def\PY@tc##1{\textcolor[rgb]{0.00,0.00,0.50}{##1}}}
\expandafter\def\csname PY@tok@go\endcsname{\def\PY@tc##1{\textcolor[rgb]{0.53,0.53,0.53}{##1}}}
\expandafter\def\csname PY@tok@gt\endcsname{\def\PY@tc##1{\textcolor[rgb]{0.00,0.27,0.87}{##1}}}
\expandafter\def\csname PY@tok@err\endcsname{\def\PY@bc##1{\setlength{\fboxsep}{0pt}\fcolorbox[rgb]{1.00,0.00,0.00}{1,1,1}{\strut ##1}}}
\expandafter\def\csname PY@tok@kc\endcsname{\let\PY@bf=\textbf\def\PY@tc##1{\textcolor[rgb]{0.00,0.50,0.00}{##1}}}
\expandafter\def\csname PY@tok@kd\endcsname{\let\PY@bf=\textbf\def\PY@tc##1{\textcolor[rgb]{0.00,0.50,0.00}{##1}}}
\expandafter\def\csname PY@tok@kn\endcsname{\let\PY@bf=\textbf\def\PY@tc##1{\textcolor[rgb]{0.00,0.50,0.00}{##1}}}
\expandafter\def\csname PY@tok@kr\endcsname{\let\PY@bf=\textbf\def\PY@tc##1{\textcolor[rgb]{0.00,0.50,0.00}{##1}}}
\expandafter\def\csname PY@tok@bp\endcsname{\def\PY@tc##1{\textcolor[rgb]{0.00,0.50,0.00}{##1}}}
\expandafter\def\csname PY@tok@fm\endcsname{\def\PY@tc##1{\textcolor[rgb]{0.00,0.00,1.00}{##1}}}
\expandafter\def\csname PY@tok@vc\endcsname{\def\PY@tc##1{\textcolor[rgb]{0.10,0.09,0.49}{##1}}}
\expandafter\def\csname PY@tok@vg\endcsname{\def\PY@tc##1{\textcolor[rgb]{0.10,0.09,0.49}{##1}}}
\expandafter\def\csname PY@tok@vi\endcsname{\def\PY@tc##1{\textcolor[rgb]{0.10,0.09,0.49}{##1}}}
\expandafter\def\csname PY@tok@vm\endcsname{\def\PY@tc##1{\textcolor[rgb]{0.10,0.09,0.49}{##1}}}
\expandafter\def\csname PY@tok@sa\endcsname{\def\PY@tc##1{\textcolor[rgb]{0.73,0.13,0.13}{##1}}}
\expandafter\def\csname PY@tok@sb\endcsname{\def\PY@tc##1{\textcolor[rgb]{0.73,0.13,0.13}{##1}}}
\expandafter\def\csname PY@tok@sc\endcsname{\def\PY@tc##1{\textcolor[rgb]{0.73,0.13,0.13}{##1}}}
\expandafter\def\csname PY@tok@dl\endcsname{\def\PY@tc##1{\textcolor[rgb]{0.73,0.13,0.13}{##1}}}
\expandafter\def\csname PY@tok@s2\endcsname{\def\PY@tc##1{\textcolor[rgb]{0.73,0.13,0.13}{##1}}}
\expandafter\def\csname PY@tok@sh\endcsname{\def\PY@tc##1{\textcolor[rgb]{0.73,0.13,0.13}{##1}}}
\expandafter\def\csname PY@tok@s1\endcsname{\def\PY@tc##1{\textcolor[rgb]{0.73,0.13,0.13}{##1}}}
\expandafter\def\csname PY@tok@mb\endcsname{\def\PY@tc##1{\textcolor[rgb]{0.40,0.40,0.40}{##1}}}
\expandafter\def\csname PY@tok@mf\endcsname{\def\PY@tc##1{\textcolor[rgb]{0.40,0.40,0.40}{##1}}}
\expandafter\def\csname PY@tok@mh\endcsname{\def\PY@tc##1{\textcolor[rgb]{0.40,0.40,0.40}{##1}}}
\expandafter\def\csname PY@tok@mi\endcsname{\def\PY@tc##1{\textcolor[rgb]{0.40,0.40,0.40}{##1}}}
\expandafter\def\csname PY@tok@il\endcsname{\def\PY@tc##1{\textcolor[rgb]{0.40,0.40,0.40}{##1}}}
\expandafter\def\csname PY@tok@mo\endcsname{\def\PY@tc##1{\textcolor[rgb]{0.40,0.40,0.40}{##1}}}
\expandafter\def\csname PY@tok@ch\endcsname{\let\PY@it=\textit\def\PY@tc##1{\textcolor[rgb]{0.25,0.50,0.50}{##1}}}
\expandafter\def\csname PY@tok@cm\endcsname{\let\PY@it=\textit\def\PY@tc##1{\textcolor[rgb]{0.25,0.50,0.50}{##1}}}
\expandafter\def\csname PY@tok@cpf\endcsname{\let\PY@it=\textit\def\PY@tc##1{\textcolor[rgb]{0.25,0.50,0.50}{##1}}}
\expandafter\def\csname PY@tok@c1\endcsname{\let\PY@it=\textit\def\PY@tc##1{\textcolor[rgb]{0.25,0.50,0.50}{##1}}}
\expandafter\def\csname PY@tok@cs\endcsname{\let\PY@it=\textit\def\PY@tc##1{\textcolor[rgb]{0.25,0.50,0.50}{##1}}}

\def\PYZbs{\char`\\}
\def\PYZus{\char`\_}
\def\PYZob{\char`\{}
\def\PYZcb{\char`\}}
\def\PYZca{\char`\^}
\def\PYZam{\char`\&}
\def\PYZlt{\char`\<}
\def\PYZgt{\char`\>}
\def\PYZsh{\char`\#}
\def\PYZpc{\char`\%}
\def\PYZdl{\char`\$}
\def\PYZhy{\char`\-}
\def\PYZsq{\char`\'}
\def\PYZdq{\char`\"}
\def\PYZti{\char`\~}
% for compatibility with earlier versions
\def\PYZat{@}
\def\PYZlb{[}
\def\PYZrb{]}
\makeatother


    % For linebreaks inside Verbatim environment from package fancyvrb. 
    \makeatletter
        \newbox\Wrappedcontinuationbox 
        \newbox\Wrappedvisiblespacebox 
        \newcommand*\Wrappedvisiblespace {\textcolor{red}{\textvisiblespace}} 
        \newcommand*\Wrappedcontinuationsymbol {\textcolor{red}{\llap{\tiny$\m@th\hookrightarrow$}}} 
        \newcommand*\Wrappedcontinuationindent {3ex } 
        \newcommand*\Wrappedafterbreak {\kern\Wrappedcontinuationindent\copy\Wrappedcontinuationbox} 
        % Take advantage of the already applied Pygments mark-up to insert 
        % potential linebreaks for TeX processing. 
        %        {, <, #, %, $, ' and ": go to next line. 
        %        _, }, ^, &, >, - and ~: stay at end of broken line. 
        % Use of \textquotesingle for straight quote. 
        \newcommand*\Wrappedbreaksatspecials {% 
            \def\PYGZus{\discretionary{\char`\_}{\Wrappedafterbreak}{\char`\_}}% 
            \def\PYGZob{\discretionary{}{\Wrappedafterbreak\char`\{}{\char`\{}}% 
            \def\PYGZcb{\discretionary{\char`\}}{\Wrappedafterbreak}{\char`\}}}% 
            \def\PYGZca{\discretionary{\char`\^}{\Wrappedafterbreak}{\char`\^}}% 
            \def\PYGZam{\discretionary{\char`\&}{\Wrappedafterbreak}{\char`\&}}% 
            \def\PYGZlt{\discretionary{}{\Wrappedafterbreak\char`\<}{\char`\<}}% 
            \def\PYGZgt{\discretionary{\char`\>}{\Wrappedafterbreak}{\char`\>}}% 
            \def\PYGZsh{\discretionary{}{\Wrappedafterbreak\char`\#}{\char`\#}}% 
            \def\PYGZpc{\discretionary{}{\Wrappedafterbreak\char`\%}{\char`\%}}% 
            \def\PYGZdl{\discretionary{}{\Wrappedafterbreak\char`\$}{\char`\$}}% 
            \def\PYGZhy{\discretionary{\char`\-}{\Wrappedafterbreak}{\char`\-}}% 
            \def\PYGZsq{\discretionary{}{\Wrappedafterbreak\textquotesingle}{\textquotesingle}}% 
            \def\PYGZdq{\discretionary{}{\Wrappedafterbreak\char`\"}{\char`\"}}% 
            \def\PYGZti{\discretionary{\char`\~}{\Wrappedafterbreak}{\char`\~}}% 
        } 
        % Some characters . , ; ? ! / are not pygmentized. 
        % This macro makes them "active" and they will insert potential linebreaks 
        \newcommand*\Wrappedbreaksatpunct {% 
            \lccode`\~`\.\lowercase{\def~}{\discretionary{\hbox{\char`\.}}{\Wrappedafterbreak}{\hbox{\char`\.}}}% 
            \lccode`\~`\,\lowercase{\def~}{\discretionary{\hbox{\char`\,}}{\Wrappedafterbreak}{\hbox{\char`\,}}}% 
            \lccode`\~`\;\lowercase{\def~}{\discretionary{\hbox{\char`\;}}{\Wrappedafterbreak}{\hbox{\char`\;}}}% 
            \lccode`\~`\:\lowercase{\def~}{\discretionary{\hbox{\char`\:}}{\Wrappedafterbreak}{\hbox{\char`\:}}}% 
            \lccode`\~`\?\lowercase{\def~}{\discretionary{\hbox{\char`\?}}{\Wrappedafterbreak}{\hbox{\char`\?}}}% 
            \lccode`\~`\!\lowercase{\def~}{\discretionary{\hbox{\char`\!}}{\Wrappedafterbreak}{\hbox{\char`\!}}}% 
            \lccode`\~`\/\lowercase{\def~}{\discretionary{\hbox{\char`\/}}{\Wrappedafterbreak}{\hbox{\char`\/}}}% 
            \catcode`\.\active
            \catcode`\,\active 
            \catcode`\;\active
            \catcode`\:\active
            \catcode`\?\active
            \catcode`\!\active
            \catcode`\/\active 
            \lccode`\~`\~ 	
        }
    \makeatother

    \let\OriginalVerbatim=\Verbatim
    \makeatletter
    \renewcommand{\Verbatim}[1][1]{%
        %\parskip\z@skip
        \sbox\Wrappedcontinuationbox {\Wrappedcontinuationsymbol}%
        \sbox\Wrappedvisiblespacebox {\FV@SetupFont\Wrappedvisiblespace}%
        \def\FancyVerbFormatLine ##1{\hsize\linewidth
            \vtop{\raggedright\hyphenpenalty\z@\exhyphenpenalty\z@
                \doublehyphendemerits\z@\finalhyphendemerits\z@
                \strut ##1\strut}%
        }%
        % If the linebreak is at a space, the latter will be displayed as visible
        % space at end of first line, and a continuation symbol starts next line.
        % Stretch/shrink are however usually zero for typewriter font.
        \def\FV@Space {%
            \nobreak\hskip\z@ plus\fontdimen3\font minus\fontdimen4\font
            \discretionary{\copy\Wrappedvisiblespacebox}{\Wrappedafterbreak}
            {\kern\fontdimen2\font}%
        }%
        
        % Allow breaks at special characters using \PYG... macros.
        \Wrappedbreaksatspecials
        % Breaks at punctuation characters . , ; ? ! and / need catcode=\active 	
        \OriginalVerbatim[#1,codes*=\Wrappedbreaksatpunct]%
    }
    \makeatother

    % Exact colors from NB
    \definecolor{incolor}{HTML}{303F9F}
    \definecolor{outcolor}{HTML}{D84315}
    \definecolor{cellborder}{HTML}{CFCFCF}
    \definecolor{cellbackground}{HTML}{F7F7F7}
    
    % prompt
    \makeatletter
    \newcommand{\boxspacing}{\kern\kvtcb@left@rule\kern\kvtcb@boxsep}
    \makeatother
    \newcommand{\prompt}[4]{
        \ttfamily\llap{{\color{#2}[#3]:\hspace{3pt}#4}}\vspace{-\baselineskip}
    }
    

    
    % Prevent overflowing lines due to hard-to-break entities
    \sloppy 
    % Setup hyperref package
    \hypersetup{
      breaklinks=true,  % so long urls are correctly broken across lines
      colorlinks=true,
      urlcolor=urlcolor,
      linkcolor=linkcolor,
      citecolor=citecolor,
      }
    % Slightly bigger margins than the latex defaults
    
    \geometry{verbose,tmargin=1in,bmargin=1in,lmargin=1in,rmargin=1in}
    
    

\begin{document}
    
    \maketitle
    \textbf{Hinweis:} Leider existieren manche der Bilder die in den Aufgaben über Links eingebunden wurden nicht
    mehr. Daher ist das Layout z.T. etwas kaputt. Ich hoffe die Lösungen sind trotzdem gut lesbar.
    
    

    
    \hypertarget{attacks-on-machine-learning}{%
\section{\texorpdfstring{\textbf{Attacks on Machine
Learning}}{Attacks on Machine Learning}}\label{attacks-on-machine-learning}}

\hypertarget{uxfcbungsaufgaben}{%
\subsection{\texorpdfstring{\textbf{Übungsaufgaben}}{Übungsaufgaben}}\label{uxfcbungsaufgaben}}

    \hypertarget{aufgabe-1---phasen-des-machine-learnings}{%
\subsection{Aufgabe 1 - Phasen des Machine
Learnings}\label{aufgabe-1---phasen-des-machine-learnings}}

\begin{quote}
In der Vorlesung wurden Ihnen die vier Phasen des Machine Learnings
vorgestellt.\\
Nennen Sie diese vier Phasen und erläutern Sie diese in ein bis zwei
Sätzen.
\end{quote}

    \begin{itemize}
\tightlist
\item
  Daten sammeln: Es muss ein entsprechend händisch annotierter Datensatz
  (für Supervised Learning) aufgebaut werden. Dabei muss vor allem auf
  die Verteilung der Daten geachtet werden
\item
  Modellauswahl: Es muss eine Architektur für das Model gewählt werden.
  Außerdem müssen Hyperparameter und die Kostenfunktion initial gewählt
  werden
\item
  Training: Das ausgewählte Modell wird mit dem ausgewählten Datensatz
  trainiert und evaluiert, eventuell müssen in diesem Schritt auch noch
  einmal Hyperparameter oder sogar das gesamte Modell angepasst werden
\item
  Nutzung: Das Modell wird als Teil einer Anwendung eingesetzt.
\end{itemize}

    \hypertarget{aufgabe-2---attacken}{%
\subsection{Aufgabe 2 - Attacken}\label{aufgabe-2---attacken}}

Im folgenden sind Ihnen drei Beispiele von Attacken auf Machine Learning
Modelle aufgelistet. Ordnen Sie den Beispielen jeweils eine vorgestellte
Attacke aus der Vorlesung zu und begründen Sie Ihre Wahl kurz.

\textbf{Beispiel 1)}\\
\textgreater{} Erkennung eines Stop-Schildes als Tempolimit-Schild.\\
%\includegraphics{https://ieeexplore.ieee.org/mediastore_new/IEEE/content/media/6287639/8600701/8685687/garg8-2909068-large.gif}\\
\textgreater{} \textgreater{}(Paper: T. Gu, K. Liu, B. Dolan-Gavitt and
S. Garg, ``BadNets: Evaluating Backdooring Attacks on Deep Neural
Networks,'' in IEEE Access, vol.~7, pp.~47230-47244, 2019, doi:
10.1109/ACCESS.2019.2909068.)

Es kann sich entweder um einen Angriff beim Ausführen handeln, also
einen targeted oder untargeted adverserial attack (da die Ziele nicht
bekannt sind kann nicht zwischen targeted und untargeted unterschieden
werden). Oder auch bereits um ein Poisioning attack, bei dem die
Trainingsdaten verändert werden.

\textbf{Beispiel 2)}\\
\textgreater{}Für dieses Beispiel schauen Sie sich bitte das
nachfolgende Video des Spiels Coast Runners 7 an.\\
\href{https://www.youtube.com/watch?time_continue=1\&v=tlOIHko8ySg}{Coast
Runners 7}

Hierbei handelt es sich höchstwahrscheinlich um Reward Hacking, das
heißt der RL-Agent optimiert zwar die Kostenfunktion erreicht dadurch
aber nicht das von den Entwicklern gewünschte Verhalten.

\textbf{Beispiel 3)}\\
\textgreater{}\includegraphics{https://d3i71xaburhd42.cloudfront.net/02bc27c39eaaa6b85d336be81b15ca19f112a950/2-Figure1-1.png}\\
\href{https://d3i71xaburhd42.cloudfront.net/02bc27c39eaaa6b85d336be81b15ca19f112a950/2-Figure1-1.png}{Bildquelle}
\textgreater{}
\textgreater{}\href{https://dl.acm.org/doi/pdf/10.1145/2810103.2813677}{Paper}

Hierbei handelt es sich um eine Angriff auf den Trainingsschritt,
ähnlich zur Membership Inference wird hier aus der Ausgabe des Modells
die (sensitive) Eingabe rekonstruiert.

    \hypertarget{aufgabe-3---fast-gradient-sign-method}{%
\subsection{Aufgabe 3 - Fast Gradient Sign
Method}\label{aufgabe-3---fast-gradient-sign-method}}

\hypertarget{a}{%
\subsubsection{a)}\label{a}}

\begin{quote}
In ihrem Paper
\href{https://ieeexplore.ieee.org/abstract/document/7958570}{Towards
Evaluating the Robustness of Neural Networks} diskutieren Carlini und
Wagner in Abschnitt II.C den Unterschied zwischen targeted Attacks und
untargeted Attacks. Handelt es sich bei der, in der Vorlesung
vorgestellten \textbf{Fast Gradient Sign} Methode, um eine targeted oder
untargeted Attack? Begründen Sie Ihre Antwort.
\end{quote}

Es handelt sich um eine untargeted Attack, da nur eine Eingabe erzeugt
wird die durch möglichst wenig Änderung den Fehler maximiert, nicht
jedoch hin zu einer bestimmten Klasse.

\hypertarget{b}{%
\subsubsection{b)}\label{b}}

\begin{quote}
Im Folgenden sind Code-Snippets gegeben, die Sie (in Google Colab oder
als Jupyter-Notebook) ausführen können. Bitte ergänzen Sie in der
entsprechend markierten Zeile (beachten Sie das TODO) den nötigen Code
zur Implementierung der \textbf{Fast Gradient Sign} Methode. Sie können
die Vorgegebenen Methoden als Hilfestellung nehmen. Eine Zelle können
Sie ausführen durch die Tastenkombination Shift+Enter.
\end{quote}

    \begin{tcolorbox}[breakable, size=fbox, boxrule=1pt, pad at break*=1mm,colback=cellbackground, colframe=cellborder]
\prompt{In}{incolor}{1}{\boxspacing}
\begin{Verbatim}[commandchars=\\\{\}]
\PY{c+c1}{\PYZsh{} Diese Zelle dient dem Import der nötigen Module. Sollten Sie das Notebook nicht in Google Colab ausführen, kann es}
\PY{c+c1}{\PYZsh{} sein, dass Sie zunächst die erforderlichen Module installieren müssen.}
\PY{k+kn}{import} \PY{n+nn}{os}
\PY{n}{os}\PY{o}{.}\PY{n}{environ}\PY{p}{[}\PY{l+s+s2}{\PYZdq{}}\PY{l+s+s2}{CUDA\PYZus{}VISIBLE\PYZus{}DEVICES}\PY{l+s+s2}{\PYZdq{}}\PY{p}{]} \PY{o}{=} \PY{l+s+s2}{\PYZdq{}}\PY{l+s+s2}{\PYZhy{}1}\PY{l+s+s2}{\PYZdq{}}
\PY{k+kn}{import} \PY{n+nn}{tensorflow} \PY{k}{as} \PY{n+nn}{tf}
\PY{k+kn}{import} \PY{n+nn}{numpy} \PY{k}{as} \PY{n+nn}{np}
\PY{k+kn}{from} \PY{n+nn}{matplotlib} \PY{k+kn}{import} \PY{n}{pyplot} \PY{k}{as} \PY{n}{plt}
\PY{k+kn}{from} \PY{n+nn}{tensorflow}\PY{n+nn}{.}\PY{n+nn}{keras}\PY{n+nn}{.}\PY{n+nn}{applications}\PY{n+nn}{.}\PY{n+nn}{mobilenet\PYZus{}v2} \PY{k+kn}{import} \PY{n}{decode\PYZus{}predictions}\PY{p}{,} \PY{n}{preprocess\PYZus{}input}
\end{Verbatim}
\end{tcolorbox}

    \begin{tcolorbox}[breakable, size=fbox, boxrule=1pt, pad at break*=1mm,colback=cellbackground, colframe=cellborder]
\prompt{In}{incolor}{2}{\boxspacing}
\begin{Verbatim}[commandchars=\\\{\}]
\PY{c+c1}{\PYZsh{} In dieser Zelle wird ein vortrainiertes Modell geladen. Dies kann je nach Internetanbindung bei lokaler Ausführung einige Zeit in Anspruch nehmen.}
\PY{n}{pretrained\PYZus{}model} \PY{o}{=} \PY{n}{tf}\PY{o}{.}\PY{n}{keras}\PY{o}{.}\PY{n}{applications}\PY{o}{.}\PY{n}{MobileNetV2}\PY{p}{(}\PY{p}{)}
\end{Verbatim}
\end{tcolorbox}

    \begin{tcolorbox}[breakable, size=fbox, boxrule=1pt, pad at break*=1mm,colback=cellbackground, colframe=cellborder]
\prompt{In}{incolor}{3}{\boxspacing}
\begin{Verbatim}[commandchars=\\\{\}]
\PY{k}{def} \PY{n+nf}{download\PYZus{}image}\PY{p}{(}\PY{n}{file\PYZus{}name}\PY{p}{,} \PY{n}{url}\PY{p}{)}\PY{p}{:}
  \PY{n}{path} \PY{o}{=} \PY{n}{tf}\PY{o}{.}\PY{n}{keras}\PY{o}{.}\PY{n}{utils}\PY{o}{.}\PY{n}{get\PYZus{}file}\PY{p}{(}\PY{n}{file\PYZus{}name}\PY{p}{,} \PY{n}{url}\PY{p}{)}
  \PY{n}{image\PYZus{}raw} \PY{o}{=} \PY{n}{tf}\PY{o}{.}\PY{n}{io}\PY{o}{.}\PY{n}{read\PYZus{}file}\PY{p}{(}\PY{n}{path}\PY{p}{)}
  \PY{n}{image} \PY{o}{=} \PY{n}{tf}\PY{o}{.}\PY{n}{image}\PY{o}{.}\PY{n}{decode\PYZus{}image}\PY{p}{(}\PY{n}{image\PYZus{}raw}\PY{p}{)}
  \PY{n}{image} \PY{o}{=} \PY{n}{tf}\PY{o}{.}\PY{n}{image}\PY{o}{.}\PY{n}{resize}\PY{p}{(}\PY{n}{image}\PY{p}{,} \PY{p}{(}\PY{l+m+mi}{224}\PY{p}{,} \PY{l+m+mi}{224}\PY{p}{)}\PY{p}{)}
  \PY{n}{image} \PY{o}{=} \PY{n}{tf}\PY{o}{.}\PY{n}{keras}\PY{o}{.}\PY{n}{applications}\PY{o}{.}\PY{n}{mobilenet\PYZus{}v2}\PY{o}{.}\PY{n}{preprocess\PYZus{}input}\PY{p}{(}\PY{n}{image}\PY{p}{)}
  \PY{n}{image} \PY{o}{=} \PY{n}{image}\PY{p}{[}\PY{k+kc}{None}\PY{p}{,} \PY{o}{.}\PY{o}{.}\PY{o}{.}\PY{p}{]}
  \PY{k}{return} \PY{n}{image}

\PY{k}{def} \PY{n+nf}{print\PYZus{}prediction}\PY{p}{(}\PY{n}{image}\PY{p}{)}\PY{p}{:}
  \PY{n}{predictions} \PY{o}{=} \PY{n}{pretrained\PYZus{}model}\PY{o}{.}\PY{n}{predict}\PY{p}{(}\PY{n}{image}\PY{p}{)}
  \PY{n}{\PYZus{}}\PY{p}{,} \PY{n}{predicted\PYZus{}label}\PY{p}{,} \PY{n}{certainty} \PY{o}{=} \PY{n}{decode\PYZus{}predictions}\PY{p}{(}\PY{n}{predictions}\PY{p}{,} \PY{n}{top}\PY{o}{=}\PY{l+m+mi}{1}\PY{p}{)}\PY{p}{[}\PY{l+m+mi}{0}\PY{p}{]}\PY{p}{[}\PY{l+m+mi}{0}\PY{p}{]}
  \PY{n+nb}{print}\PY{p}{(}\PY{l+s+s1}{\PYZsq{}}\PY{l+s+s1}{Classifier believes image to be }\PY{l+s+si}{\PYZob{}\PYZcb{}}\PY{l+s+s1}{ with certainty }\PY{l+s+si}{\PYZob{}:.2f\PYZcb{}}\PY{l+s+s1}{\PYZsq{}}\PY{o}{.}\PY{n}{format}\PY{p}{(}\PY{n}{predicted\PYZus{}label}\PY{p}{,} \PY{n}{certainty}\PY{p}{)}\PY{p}{)}

\PY{k}{def} \PY{n+nf}{show\PYZus{}image}\PY{p}{(}\PY{n}{image}\PY{p}{)}\PY{p}{:}
  \PY{c+c1}{\PYZsh{} MobileNetV2 erwartet Bilddaten im Interval [\PYZhy{}1,1] für die Visualisierung}
  \PY{c+c1}{\PYZsh{} wollen wir allerdings Bilddaten im Interval [0,1]}
  \PY{n}{plt}\PY{o}{.}\PY{n}{imshow}\PY{p}{(}\PY{n}{image}\PY{p}{[}\PY{l+m+mi}{0}\PY{p}{]}\PY{o}{*}\PY{l+m+mf}{0.5} \PY{o}{+} \PY{l+m+mf}{0.5}\PY{p}{)}
  \PY{n}{plt}\PY{o}{.}\PY{n}{show}\PY{p}{(}\PY{p}{)}
\end{Verbatim}
\end{tcolorbox}

    \begin{tcolorbox}[breakable, size=fbox, boxrule=1pt, pad at break*=1mm,colback=cellbackground, colframe=cellborder]
\prompt{In}{incolor}{4}{\boxspacing}
\begin{Verbatim}[commandchars=\\\{\}]
\PY{c+c1}{\PYZsh{} url kann durch beliebige url angepasst werden. Das Modell wurde auf quadratische Bilder trainiert.}
\PY{c+c1}{\PYZsh{} Quadratische Bilder werden in der Regel besser erkannt.}
\PY{c+c1}{\PYZsh{} Das Open Images Dataset von Google ist eine gute Quelle für Bilder, die nach label durchsucht werden können (https://storage.googleapis.com/openimages/web/index.html)}
\PY{c+c1}{\PYZsh{} Wir verwenden ein Bild von  lin padgham (https://www.flickr.com/people/linpadgham/), das wir über das Open Images Dataset gefunden haben.}
\PY{n}{fname}\PY{p}{,} \PY{n}{url} \PY{o}{=} \PY{l+s+s1}{\PYZsq{}}\PY{l+s+s1}{download.jpg}\PY{l+s+s1}{\PYZsq{}}\PY{p}{,} \PY{l+s+s1}{\PYZsq{}}\PY{l+s+s1}{https://c4.staticflickr.com/4/3130/2589169743\PYZus{}1f93f21747\PYZus{}z.jpg}\PY{l+s+s1}{\PYZsq{}}
\PY{n}{image} \PY{o}{=} \PY{n}{download\PYZus{}image}\PY{p}{(}\PY{n}{fname}\PY{p}{,} \PY{n}{url}\PY{p}{)}
\PY{n}{show\PYZus{}image}\PY{p}{(}\PY{n}{image}\PY{p}{)}
\PY{n}{print\PYZus{}prediction}\PY{p}{(}\PY{n}{image}\PY{p}{)}
\end{Verbatim}
\end{tcolorbox}

    \begin{center}
    \adjustimage{max size={0.9\linewidth}{0.9\paperheight}}{output_8_0.png}
    \end{center}
    { \hspace*{\fill} \\}
    
    \begin{Verbatim}[commandchars=\\\{\}]
Downloading data from https://storage.googleapis.com/download.tensorflow.org/dat
a/imagenet\_class\_index.json
40960/35363 [==================================] - 0s 1us/step
Classifier believes image to be king\_penguin with certainty 0.96
    \end{Verbatim}

    \begin{tcolorbox}[breakable, size=fbox, boxrule=1pt, pad at break*=1mm,colback=cellbackground, colframe=cellborder]
\prompt{In}{incolor}{5}{\boxspacing}
\begin{Verbatim}[commandchars=\\\{\}]
\PY{k}{def} \PY{n+nf}{get\PYZus{}gradient}\PY{p}{(}\PY{n}{image}\PY{p}{,} \PY{n}{label}\PY{p}{)}\PY{p}{:}
  \PY{k}{with} \PY{n}{tf}\PY{o}{.}\PY{n}{GradientTape}\PY{p}{(}\PY{p}{)} \PY{k}{as} \PY{n}{t}\PY{p}{:}
    \PY{n}{t}\PY{o}{.}\PY{n}{watch}\PY{p}{(}\PY{n}{image}\PY{p}{)}
    \PY{n}{prediction} \PY{o}{=} \PY{n}{tf}\PY{o}{.}\PY{n}{dtypes}\PY{o}{.}\PY{n}{cast}\PY{p}{(}\PY{n}{pretrained\PYZus{}model}\PY{p}{(}\PY{n}{image}\PY{p}{)}\PY{p}{[}\PY{l+m+mi}{0}\PY{p}{]}\PY{p}{,} \PY{n}{tf}\PY{o}{.}\PY{n}{float64}\PY{p}{)}
    \PY{n}{loss} \PY{o}{=} \PY{n}{tf}\PY{o}{.}\PY{n}{keras}\PY{o}{.}\PY{n}{losses}\PY{o}{.}\PY{n}{categorical\PYZus{}crossentropy}\PY{p}{(}\PY{n}{label}\PY{p}{,} \PY{n}{prediction}\PY{p}{)}
  \PY{k}{return} \PY{n}{t}\PY{o}{.}\PY{n}{gradient}\PY{p}{(}\PY{n}{loss}\PY{p}{,} \PY{n}{image}\PY{p}{)}

\PY{k}{def} \PY{n+nf}{category\PYZus{}to\PYZus{}label}\PY{p}{(}\PY{n}{imagenet\PYZus{}category}\PY{p}{)}\PY{p}{:}
  \PY{n}{label} \PY{o}{=} \PY{n}{np}\PY{o}{.}\PY{n}{zeros}\PY{p}{(}\PY{l+m+mi}{1000}\PY{p}{)}
  \PY{n}{label}\PY{p}{[}\PY{n}{imagenet\PYZus{}category}\PY{p}{]} \PY{o}{=} \PY{l+m+mi}{1}
  \PY{k}{return} \PY{n}{label}
\end{Verbatim}
\end{tcolorbox}

    \begin{tcolorbox}[breakable, size=fbox, boxrule=1pt, pad at break*=1mm,colback=cellbackground, colframe=cellborder]
\prompt{In}{incolor}{6}{\boxspacing}
\begin{Verbatim}[commandchars=\\\{\}]
\PY{c+c1}{\PYZsh{} Das label wird benötigt um den Gradienten der Kostenfunktion mit der Methode get\PYZus{}gradient zu bestimmen.}
\PY{c+c1}{\PYZsh{} Sollten Sie ein anderes Bild verwenden, dann sollten Sie auch das label anpassen. Das passende label entnehmen}
\PY{c+c1}{\PYZsh{} Sie dann bitte hier: https://gist.github.com/yrevar/942d3a0ac09ec9e5eb3a}
\PY{n}{label} \PY{o}{=} \PY{n}{category\PYZus{}to\PYZus{}label}\PY{p}{(}\PY{l+m+mi}{145}\PY{p}{)} \PY{c+c1}{\PYZsh{} 145 entspricht dem label \PYZsq{}king penguin\PYZsq{}}
\end{Verbatim}
\end{tcolorbox}

    \begin{tcolorbox}[breakable, size=fbox, boxrule=1pt, pad at break*=1mm,colback=cellbackground, colframe=cellborder]
\prompt{In}{incolor}{22}{\boxspacing}
\begin{Verbatim}[commandchars=\\\{\}]
\PY{c+c1}{\PYZsh{} TODO: Ergänzen Sie in der folgenden Zeile den Code zur Berechnung des Adversarial Images nach der Fast Gradient Sign Method (Tipp: tf.sign() verwenden)}
\PY{n}{adv\PYZus{}image} \PY{o}{=} \PY{n}{image} \PY{o}{+} \PY{l+m+mf}{0.006} \PY{o}{*} \PY{n}{tf}\PY{o}{.}\PY{n}{sign}\PY{p}{(}\PY{n}{get\PYZus{}gradient}\PY{p}{(}\PY{n}{image}\PY{p}{,} \PY{n}{label}\PY{p}{)}\PY{p}{)}
\end{Verbatim}
\end{tcolorbox}

    \begin{tcolorbox}[breakable, size=fbox, boxrule=1pt, pad at break*=1mm,colback=cellbackground, colframe=cellborder]
\prompt{In}{incolor}{20}{\boxspacing}
\begin{Verbatim}[commandchars=\\\{\}]
\PY{c+c1}{\PYZsh{} Visualisierung und Klassifizierung des Adversarial Images}
\PY{n}{show\PYZus{}image}\PY{p}{(}\PY{n}{adv\PYZus{}image}\PY{p}{)}
\PY{n}{print\PYZus{}prediction}\PY{p}{(}\PY{n}{adv\PYZus{}image}\PY{p}{)}
\end{Verbatim}
\end{tcolorbox}

    \begin{Verbatim}[commandchars=\\\{\}]
Clipping input data to the valid range for imshow with RGB data ([0..1] for
floats or [0..255] for integers).
    \end{Verbatim}

    \begin{center}
    \adjustimage{max size={0.9\linewidth}{0.9\paperheight}}{output_12_1.png}
    \end{center}
    { \hspace*{\fill} \\}
    
    \begin{Verbatim}[commandchars=\\\{\}]
Classifier believes image to be yawl with certainty 0.10
    \end{Verbatim}

    \hypertarget{c}{%
\subsubsection{c)}\label{c}}

\begin{quote}
Welche Bedeutung hat der Parameter \(\epsilon\)? Was fällt Ihnen auf,
wenn Sie einen hohen Wert für \(\epsilon\) (z.B. 1) wählen, im Vergleich
zu einem kleinen Wert (z.B. 0.000001). Verändert sich die
Klassifikation?
\end{quote}

Umso höher \(\epsilon\), umso größer ist die Pertubation. Bei kleinen
\(\epsilon\) wird die Ausgabe nur unsicherer, bei größeren \(\epsilon\)
wird die Klassikation falsch, bei sehr großen \(\epsilon\)
(\(\epsilon=1\)), ist die Pertubation auch sichtbar.

    \hypertarget{d}{%
\subsubsection{d)}\label{d}}

\begin{quote}
Verwenden Sie nun Ihre Implementierung der \textbf{Fast Gradient Sign}
Methode um ein möglichst kleines \(\epsilon\) (mit Genauigkeit von drei
Nachkommastellen) zu finden, für das eine Missklassifikation erfolgt.\\
(Tipp: Sie können eine manuelle binäre Suche durchführen, um relativ
schnell ein \(\epsilon\) finden)
\end{quote}

Ab \(\epsilon=0.006\) wird der Pinguin als ``Yawl'' klassifiziert, bei
\(\epsilon=0.005\) wird er noch korrekterweise als Pinguin
klassifiziert.

    \hypertarget{aufgabe-4---vergleich-des-carlini-wagner-angriffs-mit-der-fast-gradient-sign-methode}{%
\subsection{Aufgabe 4 - Vergleich des Carlini-Wagner Angriffs mit der
Fast Gradient Sign
Methode}\label{aufgabe-4---vergleich-des-carlini-wagner-angriffs-mit-der-fast-gradient-sign-methode}}

\begin{quote}
Vergleichen Sie, den in der Vorlesung vorgestellten Carlini-Wagner
Angriff mit der Fast Gradient Sign Methode. Welche Vor- bzw. Nachteile
bietet der Carlini-Wagner Angriff gegenüber der Fast Gradient Sign
Methode.
\end{quote}

Die Carlini-Wagner Methode ermöglicht es gezielte Angriffe
durchzuführen, d.h. der Angreifer kann gezielt bestimmen welche Klasse
er erreichen möchte, mit der Fast Gradient Sign Method wird nur eine
Missklassikation erzielt, die Klasse kann aber nicht gewählt werden. Die
Fast Gradient Sign Methode ist dafür geschlossen lösbar, im Gegensatz
zur Carlini-Wagner Methode die eine Lösung durch iterative numerische
Optimierung findet.

    \hypertarget{fazit}{%
\section{Fazit}\label{fazit}}

In dieser Übung wurden vor allem die theoretischen Konzepte aus der
Vorlesung wieder aufgegriffen und dann um eine kleine Demonstration der
Fast-Gradient-Sign Method ergänzt.

Diese Übung bestand primär aus dem Auflisten von Punkten aus der
Vorlesung und haben wenig zur Vertiefung der Themen beigetragen. Positiv
hat mir die kleine Demonstration/Implementierung der Fast-Gradient-Sign
Method gefallen.


    % Add a bibliography block to the postdoc
    
    
    
\end{document}
